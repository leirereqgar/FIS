\chapter{Glosario de términos}

\begin{itemize}

\item \textbf{Alquiler compartido:} Un alquiler puede estar a nombre de una persona, la cual convive con otras o bien puede haber un alquiler a varios nombres. 

\item \textbf{Alquiler a tiempo compartido:} El inquilino solo tiene uso y disfrute temporal durante las fechas pactadas. El interesado adquiere el derecho de pasar una temporada al año en la vivienda contratada.

\item \textbf{Alquiler vacacional:}
Viviendas alquiladas por semanas o días.

\item \textbf{Arrendador:} Aquel que cede en régimen de alquiler su inmueble mediante la firma de un contrato. 

\item \textbf{Arrendatario:} Aquel que está alguilando un inmueble.

\item \textbf{Contrato solidario:} en el que cada persona paga su parte y si una abandona, las demás deben de pagar su parte.

\item \textbf{Contrato de alquiler de vivienda:} Pacto por escrito entre inquilino y propietario en el que el inquilino adquiere el derecho de uso de la vivienda y el propietario cede en régimen de alquiler la misma. Es este pacto se establecen los derechos y deberes propios de cada parte.

\item \textbf{Desahucio:} Expulsión de los inquilinos de la propiedad debido a una demanda del arrendador por impago.

\item \textbf{Fianza de alquiler:} Cantidad de dinero al contado que se debe entregar al arrendador a la hora de formalizar el contrato como garantía de que se cumplirá con la parte.

\item \textbf{Fichero de inquilinos morosos:} Contiene información sobre arrendamientos impagados, aportada por arrendadores y gestores de arrendamientos.

\item \textbf{Impago de alquiler:} Las deudas del alquiler pueden derivar en una demanda de desahucio.

\item \textbf{Inquilino:} Al que que, mediante un contrato de arrendamiento, adquiere el derecho a usar un inmuble a cambio del pago de un canon.

\item \textbf{Mancomunado:} en el que cada persona paga solo su parte y no tienen por qué pagar la parte de otro inquilino que no lo hace. 

\item \textbf{Precio medio del alquiler:} Media de los precios de todos los alquiler de una zona.

\item \textbf{Renta:} Mensualidad que paga el inquilino al propietario por poder vivir en su vivienda.

\item \textbf{Seguro de alquiler:} Se trata de pólizas aseguradoras que proporcionan a los caseros cobertura ante el impago de las cuotas mensuales, defensa jurídica y protección contra desperfectos que puedan producirse en la vivienda.

\item \textbf{Tasador:} Se encarga de poner un precio a las viviendas según su localización, $ m^2 $, estado...

\end{itemize}