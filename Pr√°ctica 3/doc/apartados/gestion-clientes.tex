\section{Gestión de clientes}
\begin{center}
\begin{tabu}{|[2pt]p{2.5cm}|p{5cm}|p{1.5cm}|p{1.5cm}|p{1.5cm}|p{1.5cm}|[2pt]}
	\tabucline[2pt]{-}
	\textbf{Caso de uso}    & \multicolumn{4}{p{9cm}|}{\textbf{--- Nombre del CU ---}} & \multicolumn{0}{c|[2pt]}{\cellcolor{gray!25}\textbf{CU\_XX}} \\
	\tabucline[2pt]{-}
	\textbf{Actores}        & \multicolumn{5}{p{12.2cm}|[2pt]}{\small--- Listado de los actores participantes en el CU --- Podemos indicar quien es el que inicia el CU usando (I) ---} \\
	\hline
	\textbf{Tipo}           & \multicolumn{5}{p{12.2cm}|[2pt]}{\small--- Tipo del caso de uso --- Primario, Secundario u Opcional | Esencial o  Real ---} \\
	\hline
	\textbf{Referencias}    & \multicolumn{2}{p{6.5cm}|}{\small--- Indicamos qué requisitos se pueden incluir dentro de este CU ---} & \multicolumn{3}{p{4.5cm}|[2pt]}{\small--- Indicamos que requisitos se pueden incluir dentro de este CU ---} \\
	\hline
	\textbf{Precondición}   & \multicolumn{5}{p{12.2cm}|[2pt]}{\small--- Condiciones sobre el estado del sistema que tienen que ser ciertas para que se pueda realizar el CU ---} \\
	\hline
	\textbf{Postcondición}  & \multicolumn{5}{p{12.2cm}|[2pt]}{\small--- Efectos que de forma inmediata tiene la realización del CU sobre el estado del sistema ---} \\
	\hline
	\textbf{Autor}          & {\small Leire Requena Garcia} & \textbf{Fecha} & {\small 10/04/2020} & \textbf{Versión} & {\small 1.0} \\
	\tabucline[2pt]{-}
\end{tabu}

\begin{tabu}{|[2pt]p{15.68cm}|[2pt]}
	\tabucline[2pt]{-}
	\textbf{Propósito} \\
	\tabucline[2pt]{-}
	--- Descripción general del CU (Suficiente con una línea) --- \\
	\tabucline[2pt]{-}
\end{tabu}

\begin{tabu}{|[2pt]p{15.68cm}|[2pt]}
	\tabucline[2pt]{-}
	\textbf{Resumen} \\
	\tabucline[2pt]{-}
	--- Descripción de alto nivel del flujo normal (básico) del caso de uso (Suficiente con un pequeño párrafo) --- \\
	\tabucline[2pt]{-}
\end{tabu}

\begin{tabu}{|[2pt]C{0.5cm}|p{6.69cm}|C{0.5cm}|p{6.69cm}|[2pt]}
	\tabucline[2pt]{-}
	\multicolumn{4}{|[2pt]p{14.38cm}|[2pt]}{\textbf{Curso normal (Básico)}} \\
	\tabucline[2pt]{-}
	\textbf{1} & {\small Actor 1: Acción realizada por el actor} & & {\small } \\
	\hline
	\textbf{2} & {\small Actor 2: Acción realizada por el actor} & \textbf{3} & {\small Acción realizada por el sistema} \\
	\hline
	\textbf{}  & {\small } & \textbf{} & {\small }\\
	\hline
   \textbf{}  & {\small }& \textbf{N} & {\small--- Cuando se realiza la inclusión de otro caso de uso lo representamos de la forma \textit{Incluir (CU\_identificador.CU\_nombre)} ---} \\
	\hline
	\textbf{}  & {\small } & \textbf{} & {\small }\\
	\hline
	\textbf{}  & {\small Se incluyen la secuencia de acciones realizadas por los actores que intervienen en el CU\@. Se usarán frases cortas que describan el dialogo entre los actores y el sistema. Se pueden añadir referencias a elementos de un boceto del Interfaz del Usuario ---} & & {\small--- Se incluyen la secuencia de acciones que realiza el sistema ante las acciones de los actores ---} \\
	           & & & \\
	\tabucline[2pt]{-}
\end{tabu}

\begin{tabu}{|[2pt]C{0.5cm}|p{3.42cm}|p{3.92cm}|p{3.92cm}|p{3.92cm}|[2pt]}
	\tabucline[2pt]{-}
	\multicolumn{5}{|[2pt]p{15.68cm}|[2pt]}{\textbf{Cursos alternos}} \\
	\tabucline[2pt]{-}
	\textbf{1a} & \multicolumn{4}{p{14.6cm}|[2pt]}{--- Descripción de la secuencia de acciones alternas a la acción 1 del Curso Normal ---} \\
	\hline
	\multicolumn{1}{|[2pt]C{0.5cm}|}{\textbf{1b}} & \multicolumn{4}{p{14.6cm}|[2pt]}{} \\
	\hline
	\multicolumn{1}{|[2pt]C{0.5cm}|}{} & \multicolumn{4}{p{14.6cm}|[2pt]}{--- Secuencia de los cursos alternos del CU ---} \\
	\tabucline[2pt]{-}
	\multicolumn{5}{|[2pt]p{15.68cm}|[2pt]}{\textbf{Otros datos}} \\
	\tabucline[2pt]{-}
	\multicolumn{2}{|[2pt]p{3.92cm}|}{\textbf{Frecuencia esperada}} & {\small--- Número de veces que se realiza el CU por unidad de tiempo ---} & \multicolumn{1}{p{2.5cm}|}{\textbf{Rendimiento}} & {\small--- Rendimiento esperado de la secuencia de acciones del CU ---} \\
	\hline
	\multicolumn{2}{|[2pt]p{3,92cm}|}{\textbf{Importancia}} & {\small--- Importancia de este CU en el sistema (vital, alta, moderada, baja) ---} & \multicolumn{1}{p{2.5cm}|}{\textbf{Urgencia}} & {\small--- Urgencia en la realización de este CU, durante el desarrollo (alta, moderada, baja) ---} \\
	\hline
	\multicolumn{2}{|[2pt]p{3,92cm}|}{\textbf{Estado}} & {\small--- Estado actual del CU en el desarrollo ---} & \multicolumn{1}{p{2.5cm}|}{\textbf{Estabilidad}} & {\small--- Estabilidad de los requisitos asociados a este CUI (alta, moderada, baja)  ---} \\
	\tabucline[2pt]{-}
	\multicolumn{5}{|[2pt]p{15.58cm}|[2pt]}{\textbf{Comentarios}} \\
	\tabucline[2pt]{-}
	\multicolumn{5}{|[2pt]p{15.68cm}|[2pt]}{--- Comentarios adicionales sobre este CU ---} \\
	\tabucline[2pt]{-}
\end{tabu}

\end{center}

\subsection{Gestión de clientes}
\begin{center}
\begin{tabu}{|[2pt]p{2.5cm}|p{5cm}|p{1.5cm}|p{1.5cm}|p{1.5cm}|p{1.5cm}|[2pt]}
	\tabucline[2pt]{-}
	\textbf{Caso de uso}    & \multicolumn{4}{p{9cm}|}{\textbf{--- Nombre del CU ---}} & \multicolumn{0}{c|[2pt]}{\cellcolor{gray!25}\textbf{CU\_XX}} \\
	\tabucline[2pt]{-}
	\textbf{Actores}        & \multicolumn{5}{p{12.2cm}|[2pt]}{\small--- Listado de los actores participantes en el CU --- Podemos indicar quien es el que inicia el CU usando (I) ---} \\
	\hline
	\textbf{Tipo}           & \multicolumn{5}{p{12.2cm}|[2pt]}{\small--- Tipo del caso de uso --- Primario, Secundario u Opcional | Esencial o  Real ---} \\
	\hline
	\textbf{Referencias}    & \multicolumn{2}{p{6.5cm}|}{\small--- Indicamos qué requisitos se pueden incluir dentro de este CU ---} & \multicolumn{3}{p{4.5cm}|[2pt]}{\small--- Indicamos que requisitos se pueden incluir dentro de este CU ---} \\
	\hline
	\textbf{Precondición}   & \multicolumn{5}{p{12.2cm}|[2pt]}{\small--- Condiciones sobre el estado del sistema que tienen que ser ciertas para que se pueda realizar el CU ---} \\
	\hline
	\textbf{Postcondición}  & \multicolumn{5}{p{12.2cm}|[2pt]}{\small--- Efectos que de forma inmediata tiene la realización del CU sobre el estado del sistema ---} \\
	\hline
	\textbf{Autor}          & {\small Leire Requena Garcia} & \textbf{Fecha} & {\small 10/04/2020} & \textbf{Versión} & {\small 1.0} \\
	\tabucline[2pt]{-}
\end{tabu}

\begin{tabu}{|[2pt]p{15.68cm}|[2pt]}
	\tabucline[2pt]{-}
	\textbf{Propósito} \\
	\tabucline[2pt]{-}
	--- Descripción general del CU (Suficiente con una línea) --- \\
	\tabucline[2pt]{-}
\end{tabu}

\begin{tabu}{|[2pt]p{15.68cm}|[2pt]}
	\tabucline[2pt]{-}
	\textbf{Resumen} \\
	\tabucline[2pt]{-}
	--- Descripción de alto nivel del flujo normal (básico) del caso de uso (Suficiente con un pequeño párrafo) --- \\
	\tabucline[2pt]{-}
\end{tabu}

\begin{tabu}{|[2pt]C{0.5cm}|p{6.69cm}|C{0.5cm}|p{6.69cm}|[2pt]}
	\tabucline[2pt]{-}
	\multicolumn{4}{|[2pt]p{14.38cm}|[2pt]}{\textbf{Curso normal (Básico)}} \\
	\tabucline[2pt]{-}
	\textbf{1} & {\small Actor 1: Acción realizada por el actor} & & {\small } \\
	\hline
	\textbf{2} & {\small Actor 2: Acción realizada por el actor} & \textbf{3} & {\small Acción realizada por el sistema} \\
	\hline
	\textbf{}  & {\small } & \textbf{} & {\small }\\
	\hline
   \textbf{}  & {\small }& \textbf{N} & {\small--- Cuando se realiza la inclusión de otro caso de uso lo representamos de la forma \textit{Incluir (CU\_identificador.CU\_nombre)} ---} \\
	\hline
	\textbf{}  & {\small } & \textbf{} & {\small }\\
	\hline
	\textbf{}  & {\small Se incluyen la secuencia de acciones realizadas por los actores que intervienen en el CU\@. Se usarán frases cortas que describan el dialogo entre los actores y el sistema. Se pueden añadir referencias a elementos de un boceto del Interfaz del Usuario ---} & & {\small--- Se incluyen la secuencia de acciones que realiza el sistema ante las acciones de los actores ---} \\
	           & & & \\
	\tabucline[2pt]{-}
\end{tabu}

\begin{tabu}{|[2pt]C{0.5cm}|p{3.42cm}|p{3.92cm}|p{3.92cm}|p{3.92cm}|[2pt]}
	\tabucline[2pt]{-}
	\multicolumn{5}{|[2pt]p{15.68cm}|[2pt]}{\textbf{Cursos alternos}} \\
	\tabucline[2pt]{-}
	\textbf{1a} & \multicolumn{4}{p{14.6cm}|[2pt]}{--- Descripción de la secuencia de acciones alternas a la acción 1 del Curso Normal ---} \\
	\hline
	\multicolumn{1}{|[2pt]C{0.5cm}|}{\textbf{1b}} & \multicolumn{4}{p{14.6cm}|[2pt]}{} \\
	\hline
	\multicolumn{1}{|[2pt]C{0.5cm}|}{} & \multicolumn{4}{p{14.6cm}|[2pt]}{--- Secuencia de los cursos alternos del CU ---} \\
	\tabucline[2pt]{-}
	\multicolumn{5}{|[2pt]p{15.68cm}|[2pt]}{\textbf{Otros datos}} \\
	\tabucline[2pt]{-}
	\multicolumn{2}{|[2pt]p{3.92cm}|}{\textbf{Frecuencia esperada}} & {\small--- Número de veces que se realiza el CU por unidad de tiempo ---} & \multicolumn{1}{p{2.5cm}|}{\textbf{Rendimiento}} & {\small--- Rendimiento esperado de la secuencia de acciones del CU ---} \\
	\hline
	\multicolumn{2}{|[2pt]p{3,92cm}|}{\textbf{Importancia}} & {\small--- Importancia de este CU en el sistema (vital, alta, moderada, baja) ---} & \multicolumn{1}{p{2.5cm}|}{\textbf{Urgencia}} & {\small--- Urgencia en la realización de este CU, durante el desarrollo (alta, moderada, baja) ---} \\
	\hline
	\multicolumn{2}{|[2pt]p{3,92cm}|}{\textbf{Estado}} & {\small--- Estado actual del CU en el desarrollo ---} & \multicolumn{1}{p{2.5cm}|}{\textbf{Estabilidad}} & {\small--- Estabilidad de los requisitos asociados a este CUI (alta, moderada, baja)  ---} \\
	\tabucline[2pt]{-}
	\multicolumn{5}{|[2pt]p{15.58cm}|[2pt]}{\textbf{Comentarios}} \\
	\tabucline[2pt]{-}
	\multicolumn{5}{|[2pt]p{15.68cm}|[2pt]}{--- Comentarios adicionales sobre este CU ---} \\
	\tabucline[2pt]{-}
\end{tabu}

\end{center}
