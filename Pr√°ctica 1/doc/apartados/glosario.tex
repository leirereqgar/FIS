Alquiler compartido:
Un alquiler puede estar a nombre de una persona, la cual convive con otras o bien puede haber un alquiler a varios nombres. 
    • Contrato solidario: en el que cada persona paga su parte y si una abandona, las demás deben de pagar su parte
    • Mancomunado: en el que cada persona paga solo su parte y no tienen por qué pagar la parte de otro inquilino que no lo hace.

Alquiler a tiempo compartido:
El inquilino solo tiene uso y disfrute temporal durante las fechas pactadas. El interesado adquiere el derecho de pasar una temporada al año en la vivienda contratada.

Alquiler vacacional:
Viviendas alquiladas por semanas o días. 

Arrendador:
Aquel que cede en régimen de alquiler su inmueble mediante la firma de un contrato. 

Inquilino:
Al que que, mediante un contrato de arrendamiento, adquiere el derecho a usar un inmuble a cambio del pago de un canon.

Contrato de alquiler de vivienda:
Pacto por escrito entre inquilino y propietario en el que el inquilino adquiere el derecho de uso de la vivienda y el propietario cede en régimen de alquiler la misma. Es este pacto se establecen los derechos y deberes propios de cada parte.

Fianza de alquiler: Cantidad de dinero al contado que se debe entregar al arrendador a la hora de formalizar el contrato como garantía de que se cumplirá con la parte.

Renta: 
Mensualidad que paga el inquilino al propietario por poder vivir en su vivienda.

Fichero de inquilinos morosos:
Contiene información sobre arrendamientos impagados, aportada por arrendadores y gestores de arrendamientos.

Impago de alquiler:
Las deudas del alquiler pueden derivar en una demanda de desahucio.

Desahucio:
Expulsión de los inquilinos de la propiedad debido a una demanda del arrendador por impago.

Precio medio del alquiler:
Media de los precios de todos los alquiler de una zona.

Seguro de alquiler:
Se trata de pólizas aseguradoras que proporcionan a los caseros cobertura ante el impago de las cuotas mensuales, defensa jurídica y protección contra desperfectos que puedan producirse en la vivienda.
