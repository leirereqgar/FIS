\section{Gestión de alquileres}
\subsection{\texttt{consultarContratoAlquiler:}}

\begin{center}
\begin{tabular}{l p{13cm}}
\textbf{Nombre}          & \code{consultarContratoAlquiler (idContratoAlquiler)} \\
\midrule
\textbf{Responsabilidad} & Consulta el contrato del alquiler identificado con idContratoAlquiler \\
\textbf{Tipo}            & GestionAlquileres                                \\
\textbf{Notas}           & Se podrán consultar todos los aspectos relacionados con ese contrato                                  \\
\textbf{Excepciones}     &                                  \\
\textbf{Salida}          & infoContratoAlquiler                                  \\
\textbf{Precondiciones}  & Existe un objeto contratoAlquiler para idContratoAlquiler                                   \\
\textbf{Postcondiciones} & Se han devuelto los datos referentes al objeto contratoAlquiler                                  \\
\textbf{Autor}           & Alejandro Molina Criado                                  \\
\end{tabular}
\end{center}

\subsection{\texttt{altaAveria:}}
\begin{center}
\begin{tabular}{l p{13cm}}
\textbf{Nombre}          & \code{altaAveria (idViviendaAlquiler , descripcionAveria , fechaNotificacion)} \\
\midrule
\textbf{Responsabilidad} & Registrar una avería en el sistema                                 \\
\textbf{Tipo}            & GestionAlquileres                                   \\
\textbf{Notas}           & La descripción de la avería debe ser breve y clara                                    \\
\textbf{Excepciones}     &                                    \\
\textbf{Salida}          & Confirmación de registro de la avería en el sistema                                  \\
\textbf{Precondiciones}  &                                    \\
\textbf{Postcondiciones} & Se creó un objeto de la clase Averia                                     \\
\textbf{Autor}           & Alejandro Molina Criado                                 \\
\end{tabular}
\end{center}

\subsection{\texttt{altaContratoAlquiler:}}
\begin{center}
\begin{tabular}{l p{13cm}}
\textbf{Nombre}          & \code{altaContratoAlquiler (idViviendaAlquiler , idInquilino , fechaInicioContrato , fechaFinContrato , fianza , cuotaMensual)} \\
\midrule
\textbf{Responsabilidad} & Registrar un nuevo contrato de Alquiler en el sistema                                   \\
\textbf{Tipo}            & GestionAlquileres\\
\textbf{Notas}           &                                    \\
\textbf{Excepciones}     &                                    \\
\textbf{Salida}          & Confirmación de registro del contrato de Alquiler en el sistema                                      \\
\textbf{Precondiciones}  & El contrato a registrar cumpla la legislación vigente                                  \\
\textbf{Postcondiciones} & Se creó un objeto de la clase ContratoAlquiler                                 \\
\textbf{Autor}           & Alejandro Molina Criado                  \\
\end{tabular}
\end{center}

\subsection{\texttt{bajaContratoAlquiler:}}
\begin{center}
\begin{tabular}{l p{13cm}}
\textbf{Nombre}          & \code{altaContratoAlquiler (idContratoAlquiler)} \\
\midrule
\textbf{Responsabilidad} & Eliminar el contrato de Alquiler en el sistema                                   \\
\textbf{Tipo}            & GestionAlquileres                                    \\
\textbf{Notas}           &                                    \\
\textbf{Excepciones}     &                                    \\
\textbf{Salida}          &                                       \\
\textbf{Precondiciones}  & El objeto ContratoAlquiler identificado por idContratoAlquiler existe\\
\textbf{Postcondiciones} & Se eliminó un objeto de la clase ContratoAlquiler identificado por idContratoAlquiler                \\
\textbf{Autor}           & Alejandro Molina Criado                                       \\
\end{tabular}
\end{center}

\subsection{\texttt{altaInquilino:}}
\begin{center}
\begin{tabular}{l p{13cm}}
\textbf{Nombre}          & \code{altaInquilino (dni , nombre , apellidos , direccion , numeroCuentaBanca)} \\
\midrule
\textbf{Responsabilidad} &                                    \\
\textbf{Tipo}            & GestionAlquileres                                     \\
\textbf{Notas}           & Todos los datos almacenados referentes al inquilino deben cumplir la ley de protección de datos                                  \\
\textbf{Excepciones}     &                                    \\
\textbf{Salida}          & Confirmación de que el inquilino ha sido dado de alta     \\
\textbf{Precondiciones}  &                                   \\
\textbf{Postcondiciones} & Se ha creado un objeto Inquilino identificado por idInquilino                       \\
\textbf{Autor}           & Alejandro Molina Criado\\
\end{tabular}
\end{center}


\subsection{\texttt{bajaInquilino:}}
\begin{center}
\begin{tabular}{l p{13cm}}
\textbf{Nombre}          & \code{bajaInquilino (idInquilino)} \\
\midrule
\textbf{Responsabilidad} & Eliminar del sistema de inquilinos activos al inquilino identificado por idInquilino                                   \\
\textbf{Tipo}            & GestionAlquileres\\
\textbf{Notas}           & Al eliminar un inquilino , este debería volver a darse de alta en el sistema en caso de que volver a alquilar                    \\
\textbf{Excepciones}     & No existe ningun objeto inquilino identificado con idInquilino                   \\
\textbf{Salida}          &          \\
\textbf{Precondiciones}  & Existe un objeto Inquilino identificado por idInquilino                                   \\
\textbf{Postcondiciones} & Se ha eliminado el objeto de la clase Inquilino identificado con idInquilino                                  \\
\textbf{Autor}           & Alejandro Molina Criado                                     \\
\end{tabular}
\end{center}

\subsection{\texttt{devolverFianzaInquilino:}}
\begin{center}
\begin{tabular}{l p{13cm}}
\textbf{Nombre}          & \code{devolverFianzaInquilino (idContratoAlquiler , fecha)} \\
\midrule
\textbf{Responsabilidad} & Se devuelve la fianza al inquilino del ContratoAlquiler                                   \\
\textbf{Tipo}            & GestionAlquileres                               \\
\textbf{Notas}           &                                    \\
\textbf{Excepciones}     &                                    \\
\textbf{Salida}          &                                    \\
\textbf{Precondiciones}  & El inquilino no debe de haber recibido la fianza                                 \\
\textbf{Postcondiciones} & Se ha devuelto la fianza al inquilino                                   \\
\textbf{Autor}           & Alejandro Molina Criado                      \\
\end{tabular}
\end{center}

\subsection{\texttt{pagarReciboMetalico:}}
\begin{center}
\begin{tabular}{l p{13cm}}
\textbf{Nombre}          & \code{pagarReciboMetalico (idInquilino , idContratoAlquiler , mensualidad , fecha)} \\
\midrule
\textbf{Responsabilidad} & Pagar un recibo en metálico     \\
\textbf{Tipo}            & GestionAlquileres                                     \\
\textbf{Notas}           &                                    \\
\textbf{Excepciones}     &                                    \\
\textbf{Salida}          & Recibos = lista de \code{\{idCliente,idViviendaAlquiler,mensualidad,fecha\}}                                   \\
\textbf{Precondiciones}  & El objeto reciboAlquiler debe de existir                                   \\
\textbf{Postcondiciones} & Devolvió en recibos una lista con la identificación del objeto de ViviendaAlquiler, la identificación del objeto de Cliente y mensualidad    \\
\textbf{Autor}           & Alejandro Molina Criado                                 \\
\end{tabular}
\end{center}

\subsection{\texttt{añadirPagoCuentaCliente:}}
\begin{center}
\begin{tabular}{l p{13cm}}
\textbf{Nombre}          & \code{añadirPagoCuentaCliente (idCliente , idContratoAlquiler , mensualidad , fecha)} \\
\midrule
\textbf{Responsabilidad} & Añadir un pago realizado o recibido a la lista de pagos de la cuenta de un cliente                                   \\
\textbf{Tipo}            & GestionAlquileres\\
\textbf{Notas}           &                                    \\
\textbf{Excepciones}     &                                    \\
\textbf{Salida}          & listaPagos                                   \\
\textbf{Precondiciones}  & Deben existir los objetos cliente y contrato identificados por idCliente e idContratoAlquiler                                   \\
\textbf{Postcondiciones} & Se añadió un pago a la lista de pagos de la cuenta del cliente identificado por idCliente                                   \\
\textbf{Autor}           & Alejandro Molina Criado                                   \\
\end{tabular}
\end{center}

\subsection{\texttt{obtenerRecibosDevueltos:}}
\begin{center}
\begin{tabular}{l p{13cm}}
\textbf{Nombre}          & \code{obtenerRecibosDevueltos (idInquilino , fecha)} \\
\midrule
\textbf{Responsabilidad} & Generar una lista visualizable de los recibos que han sido devueltos                                   \\
\textbf{Tipo}            & GestionAlquileres\\
\textbf{Notas}           & La lista debe de contener todos los datos de dichos recibos                                   \\
\textbf{Excepciones}     & No existen recibos devueltos                                   \\
\textbf{Salida}          & listaRecibosDevueltos                              \\
\textbf{Precondiciones}  & El objeto inquilino identificado por idInquilino debe existir . La fecha debe coincidir con la fecha de devolución\\
\textbf{Postcondiciones} & Se generó una lista con los recibos del sistema que aparecen como devueltos                                   \\
\textbf{Autor}           & Alejandro Molina Criado  \\
\end{tabular}
\end{center}

\subsection{\texttt{obtenerRecibosPagados:}}
\begin{center}
\begin{tabular}{l p{13cm}}
\textbf{Nombre}          & \code{obtenerRecibosPagados (idInquilino , fecha)} \\
\midrule
\textbf{Responsabilidad} & Generar una lista visualizable de los recibos que han sido pagados                                   \\
\textbf{Tipo}            & GestionAlquileres\\
\textbf{Notas}           & La lista debe de contener todos los datos de dichos recibos                                   \\
\textbf{Excepciones}     & No existen recibos pagados   \\
\textbf{Salida}          & listaRecibosPagados                               \\
\textbf{Precondiciones}  & El objeto inquilino identificado por idInquilino debe existir . La fecha debe coincidir con la fecha de pago\\
\textbf{Postcondiciones} & Se generó una lista con los recibos del sistema que aparecen como pagados   \\
\textbf{Autor}           & Alejandro Molina Criado\\
\end{tabular}
\end{center}

\subsection{\texttt{iniciarTramiteMoroso:}}
\begin{center}
    \begin{tabular}{l p{13cm}}
    \textbf{Nombre}          & \code{iniciarTramiteMoroso (idInquilino, listaIdRecibosDevueltos,fecha)} \\
    \midrule
    \textbf{Responsabilidad} &  Permite iniciar un tramite de morosidad                                \\
    \textbf{Tipo}            &  GestionAlquileres                                  \\
    \textbf{Notas}           &                                    \\
    \textbf{Excepciones}     &                                    \\
    \textbf{Salida}          &                                    \\
    \textbf{Precondiciones}  &                                   \\
    \textbf{Postcondiciones} & El objeto inquilino con idInquilino debe haber recibido un lista de todos los obejetos RecibosAlquiler que figuren como sin ser pagados                                  \\
    \textbf{Autor}           & Rafael Guzman Valverde                                   \\
    \end{tabular}
    \end{center}

\subsection{\texttt{generarRecibosAlquiler:}}
\begin{center}
    \begin{tabular}{l p{13cm}}
    \textbf{Nombre}          & \code{generarRecibosAlquiler (idRecibo)} \\
    \midrule
    \textbf{Responsabilidad} &  Permite generar todos los recibos de un alquiler respecto de un cliente                                \\
    \textbf{Tipo}            &  GestionAlquileres                                  \\
    \textbf{Notas}           &                                    \\
    \textbf{Excepciones}     & El cliente no tiene alquileres                                   \\
    \textbf{Salida}          & listaIdRecibos = una lista de objetos de RecibosAlquiler                                   \\
    \textbf{Precondiciones}  &                                    \\
\multirow{2}{*}{\textbf{Postcondiciones}} & Se debe haber calculado la mensualidad para cada recibo\\
											& Se debe haber generado una lista con todos los alquileres que tiene el cliente                                 \\
    \textbf{Autor}           & Rafael Guzman Valverde                                   \\
    \end{tabular}
    \end{center}

\subsection{\texttt{enviarReciboBanco:}}
\begin{center}
\begin{tabular}{l p{13cm}}
\textbf{Nombre}          & \code{enviarReciboBanco (idRecibo)} \\
\midrule
\textbf{Responsabilidad} &  Envia el recibo al banco                                  \\
\textbf{Tipo}            &  GestionAlquileres                                  \\
\textbf{Notas}           &                                    \\
\textbf{Excepciones}     &                                    \\
\textbf{Salida}          &                                    \\
\textbf{Precondiciones}  &                                    \\
\textbf{Postcondiciones} & Se envió satisfactoriamente el recibo al banco                                   \\
\textbf{Autor}           & Rafael Guzman Valverde                                   \\
\end{tabular}
\end{center}