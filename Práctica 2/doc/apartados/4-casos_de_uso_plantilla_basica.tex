\chapter{Descripción general de los casos de uso}
\section{Gestión de usuarios}
\subsection{Dar de alta un nuevo usuario}\label{CU-4.1.1}
\begin{center}
\begin{tabu}{|[2pt]p{2.5cm}|p{5cm}|p{1.5cm}|p{1.5cm}|p{1.5cm}|p{1.5cm}|[2pt]}
	\tabucline[2pt]{-}
	\textbf{Caso de uso}    & \multicolumn{4}{p{9cm}|}{\textbf{Dar de alta a un nuevo usuario}} & \multicolumn{0}{c|[2pt]}{\cellcolor{gray!25}\textbf{CU\_01}} \\
	\hline
	\textbf{Actores}        & \multicolumn{5}{l|[2pt]}{Arrendador, arrendatario} \\
	\hline
	\textbf{Tipo}           & \multicolumn{5}{l|[2pt]}{Primario y esencial} \\
	\hline
	\textbf{Referencias}    & \multicolumn{2}{l|}{RF-1.1} & \multicolumn{3}{l|[2pt]}{} \\
	\hline
	\textbf{Precondición}   & \multicolumn{5}{l|[2pt]}{} \\
	\hline
	\textbf{Postcondición}  & \multicolumn{5}{l|[2pt]}{Los datos del usuario se quedan guardados en la base de datos} \\
	\hline
	\textbf{Autor}          & Leire Requena Garcia & \textbf{Fecha} & 28/03/20 & \textbf{Versión} & 1.0 \\
	\tabucline[2pt]{-}
\end{tabu}

\begin{tabu}{|[2pt]p{15.68cm}|[2pt]}
	\tabucline[2pt]{-}
	\textbf{Propósito} \\
	\hline
	Permitir el registro de nuevos usuarios de cualquier tipo\\
	\tabucline[2pt]{-}
\end{tabu}

\begin{tabu}{|[2pt]p{15.68cm}|[2pt]}
	\tabucline[2pt]{-}
	\textbf{Resumen} \\
	\hline
	El usuario introduce sus datos personales y el sistema crea un nuevo registro en la base de datos con ellos, y le da acceso al usuario.\\
	\tabucline[2pt]{-}
\end{tabu}
\end{center}

\subsection{Consulta de un usuario}\label{CU-4.1.2}

\begin{center}
\begin{tabu}{|[2pt]p{2.5cm}|p{5cm}|p{1.5cm}|p{1.5cm}|p{1.5cm}|p{1.5cm}|[2pt]}
	\tabucline[2pt]{-}
	\textbf{Caso de uso}    & \multicolumn{4}{p{9cm}|}{\textbf{Consulta de un usuario}} & \multicolumn{0}{c|[2pt]}{\cellcolor{gray!25}\textbf{CU\_02}} \\
	\hline
	\textbf{Actores}        & \multicolumn{5}{l|[2pt]}{Arrendador, arrendatario, agente inmobiliario} \\
	\hline
	\textbf{Tipo}           & \multicolumn{5}{l|[2pt]}{Primario y esencial} \\
	\hline
	\textbf{Referencias}    & \multicolumn{2}{l|}{RF-1.1} & \multicolumn{3}{l|[2pt]}{} \\
	\hline
	\textbf{Precondición}   & \multicolumn{5}{l|[2pt]}{Debe existir el usuario en la base de datos} \\
	\hline
	\textbf{Postcondición}  & \multicolumn{5}{l|[2pt]}{Se obtiene la información del usuario} \\
	\hline
	\textbf{Autor}          & Leire Requena Garcia & \textbf{Fecha} & 28/03/20 & \textbf{Versión} & 1.0 \\
	\tabucline[2pt]{-}
\end{tabu}

\begin{tabu}{|[2pt]p{15.68cm}|[2pt]}
	\tabucline[2pt]{-}
	\textbf{Propósito} \\
	\hline
	Consultar los datos de un usuario \\
	\tabucline[2pt]{-}
\end{tabu}

\begin{tabu}{|[2pt]p{15.68cm}|[2pt]}
	\tabucline[2pt]{-}
	\textbf{Resumen} \\
	\hline
	El interesado accede a la información de un usuario dado de alta en el sistema, como puede ser su historial, la lista de citas con propietarios o agentes inmobiliarios... \\
	\tabucline[2pt]{-}
\end{tabu}
\end{center}

\subsection{Modificación de un usuario}\label{CU-4.1.3}

\begin{center}
\begin{tabu}{|[2pt]p{2.5cm}|p{5cm}|p{1.5cm}|p{1.5cm}|p{1.5cm}|p{1.5cm}|[2pt]}
	\tabucline[2pt]{-}
	\textbf{Caso de uso}    & \multicolumn{4}{p{9cm}|}{\textbf{Modificación de un usuario}} & \multicolumn{0}{c|[2pt]}{\cellcolor{gray!25}\textbf{CU\_03}} \\
	\hline
	\textbf{Actores}        & \multicolumn{5}{l|[2pt]}{Arrendador, arrendatario} \\
	\hline
	\textbf{Tipo}           & \multicolumn{5}{l|[2pt]}{Primario y esencial} \\
	\hline
	\textbf{Referencias}    & \multicolumn{2}{l|}{RF-3.3} & \multicolumn{3}{l|[2pt]}{} \\
	\hline
	\textbf{Precondición}   & \multicolumn{5}{l|[2pt]}{El usuario debe estar dado de alta con anterioridad} \\
	\hline
	\textbf{Postcondición}  & \multicolumn{5}{l|[2pt]}{Se modifica la información relativa en la base de datos} \\
	\hline
	\textbf{Autor}          & Leire Requena Garcia & \textbf{Fecha} & 28/03/20 & \textbf{Versión} & 1.0 \\
	\tabucline[2pt]{-}
\end{tabu}

\begin{tabu}{|[2pt]p{15.68cm}|[2pt]}
	\tabucline[2pt]{-}
	\textbf{Propósito} \\
	\hline
	Modificar la información relativa a los datos personales de ese usuario \\
	\tabucline[2pt]{-}
\end{tabu}

\begin{tabu}{|[2pt]p{15.68cm}|[2pt]}
	\tabucline[2pt]{-}
	\textbf{Resumen} \\
	\hline
	El interesado accede al sistema para modificar la información relativa \\
	\tabucline[2pt]{-}
\end{tabu}
\end{center}

\subsection{Baja de usuario}\label{CU-4.1.4}

\begin{center}
\begin{tabu}{|[2pt]p{2.5cm}|p{5cm}|p{1.5cm}|p{1.5cm}|p{1.5cm}|p{1.5cm}|[2pt]}
	\tabucline[2pt]{-}
	\textbf{Caso de uso}    & \multicolumn{4}{p{9cm}|}{\textbf{Baja de un usuario}} & \multicolumn{0}{c|[2pt]}{\cellcolor{gray!25}\textbf{CU\_04}} \\
	\hline
	\textbf{Actores}        & \multicolumn{5}{l|[2pt]}{Arrendatario, arrendador y agente inmobiliario} \\
	\hline
	\textbf{Tipo}           & \multicolumn{5}{l|[2pt]}{Primario y esencial} \\
	\hline
	\textbf{Referencias}    & \multicolumn{2}{l|}{RF-1.2} & \multicolumn{3}{l|[2pt]}{} \\
	\hline
	\textbf{Precondición}   & \multicolumn{5}{l|[2pt]}{El usuario debe estar dado de alta en el sistema} \\
	\hline
	\textbf{Postcondición}  & \multicolumn{5}{l|[2pt]}{El perfil desaparecerá de las bases de datos} \\
	\hline
	\textbf{Autor}          & Leire Requena Garcia & \textbf{Fecha} & 28/03/20 & \textbf{Versión} & 1.0 \\
	\tabucline[2pt]{-}
\end{tabu}

\begin{tabu}{|[2pt]p{15.68cm}|[2pt]}
	\tabucline[2pt]{-}
	\textbf{Propósito} \\
	\hline
	Eliminar del sistema el perfil del usuario y sus datos asociados \\
	\tabucline[2pt]{-}
\end{tabu}

\begin{tabu}{|[2pt]p{15.68cm}|[2pt]}
	\tabucline[2pt]{-}
	\textbf{Resumen} \\
	\hline
	Permite dar de baja a un usuario, esta acción la puede hacer el usuario, ya sea arrendador o arrendatario o un agente inmobiliario \\
	\tabucline[2pt]{-}
\end{tabu}
\end{center}

\section{Gestión de viviendas}
\subsection{Alta de vivienda}\label{CU-4.2.1}
\begin{center}
\begin{tabu}{|[2pt]p{2.5cm}|p{5cm}|p{1.5cm}|p{1.5cm}|p{1.5cm}|p{1.5cm}|[2pt]}
	\tabucline[2pt]{-}
	\textbf{Caso de uso}    & \multicolumn{4}{p{9cm}|}{\textbf{Alta de vivienda}} & \multicolumn{0}{c|[2pt]}{\cellcolor{gray!25}\textbf{CU\_05}} \\
	\hline
	\textbf{Actores}        & \multicolumn{5}{l|[2pt]}{Arrendador} \\
	\hline
	\textbf{Tipo}           & \multicolumn{5}{l|[2pt]}{Primario y esencial} \\
	\hline
	\textbf{Referencias}    & \multicolumn{2}{l|}{RF-2.1} & \multicolumn{3}{l|[2pt]}{} \\
	\hline
	\textbf{Precondición}   & \multicolumn{5}{l|[2pt]}{El usuario debe estar registrado en el sistema} \\
	\hline
	\textbf{Postcondición}  & \multicolumn{5}{l|[2pt]}{La información de la vivienda pasa a estar en la base de datos} \\
	\hline
	\textbf{Autor}          & Rafael Guzmán Valverde & \textbf{Fecha} & 28/03/20 & \textbf{Versión} & 1.0 \\
	\tabucline[2pt]{-}
\end{tabu}

\begin{tabu}{|[2pt]p{15.68cm}|[2pt]}
	\tabucline[2pt]{-}
	\textbf{Propósito} \\
	\hline
	Introducir una nueva vivienda en la base de datos \\
	\tabucline[2pt]{-}
\end{tabu}

\begin{tabu}{|[2pt]p{15.68cm}|[2pt]}
	\tabucline[2pt]{-}
	\textbf{Resumen} \\
	\hline
	Se introduce toda la información de la vivienda y pasa a estar dentro de la base de datos \\
	\tabucline[2pt]{-}
\end{tabu}
\end{center}

\subsection{Modificación de vivienda}\label{CU-4.2.2}
\begin{center}
\begin{tabu}{|[2pt]p{2.5cm}|p{5cm}|p{1.5cm}|p{1.5cm}|p{1.5cm}|p{1.5cm}|[2pt]}
	\tabucline[2pt]{-}
	\textbf{Caso de uso}    & \multicolumn{4}{p{9cm}|}{\textbf{Modificación de vivienda}} & \multicolumn{0}{c|[2pt]}{\cellcolor{gray!25}\textbf{CU\_06}} \\
	\hline
	\textbf{Actores}        & \multicolumn{5}{l|[2pt]}{Arrendador, tasador} \\
	\hline
	\textbf{Tipo}           & \multicolumn{5}{l|[2pt]}{Primario y esencial} \\
	\hline
	\textbf{Referencias}    & \multicolumn{2}{l|}{RF-2.4} & \multicolumn{3}{l|[2pt]}{} \\
	\hline
	\textbf{Precondición}   & \multicolumn{5}{l|[2pt]}{La vivienda debe estar con anterioridad en el sistema} \\
	\hline
	\textbf{Postcondición}  & \multicolumn{5}{l|[2pt]}{LA información relativa a la vivienda cambia} \\
	\hline
	\textbf{Autor}          & Rafael Guzmán Valverde & \textbf{Fecha} & 28/03/20 & \textbf{Versión} & 1.0 \\
	\tabucline[2pt]{-}
\end{tabu}

\begin{tabu}{|[2pt]p{15.68cm}|[2pt]}
	\tabucline[2pt]{-}
	\textbf{Propósito} \\
	\hline
	Modificar la información relativa a la vivienda \\
	\tabucline[2pt]{-}
\end{tabu}

\begin{tabu}{|[2pt]p{15.68cm}|[2pt]}
	\tabucline[2pt]{-}
	\textbf{Resumen} \\
	\hline
	El tasador podrá poner el precio una vez cuando visite la vivienda y el propietario podrá marcar la vivienda como de intercambio o bajar y subir el precio si lo ve necesario \\
	\tabucline[2pt]{-}
\end{tabu}
\end{center}

\subsection{Consulta de vivienda}\label{CU-4.2.3}
\begin{center}
\begin{tabu}{|[2pt]p{2.5cm}|p{5cm}|p{1.5cm}|p{1.5cm}|p{1.5cm}|p{1.5cm}|[2pt]}
	\tabucline[2pt]{-}
	\textbf{Caso de uso}    & \multicolumn{4}{p{9cm}|}{\textbf{Consulta de vivienda}} & \multicolumn{0}{c|[2pt]}{\cellcolor{gray!25}\textbf{CU\_07}} \\
	\hline
	\textbf{Actores}        & \multicolumn{5}{l|[2pt]}{Arrendador, arrendatario y agente inmobiliario} \\
	\hline
	\textbf{Tipo}           & \multicolumn{5}{l|[2pt]}{Primario y esencial} \\
	\hline
	\textbf{Referencias}    & \multicolumn{2}{l|}{RF-2.3} & \multicolumn{3}{l|[2pt]}{} \\
	\hline
	\textbf{Precondición}   & \multicolumn{5}{l|[2pt]}{La vivienda debe estar en el sistema con anterioridad} \\
	\hline
	\textbf{Postcondición}  & \multicolumn{5}{l|[2pt]}{Se accede a la información de la vivienda} \\
	\hline
	\textbf{Autor}          & Rafael Guzmán Valverde & \textbf{Fecha} & 28/03/20 & \textbf{Versión} & 1.0 \\
	\tabucline[2pt]{-}
\end{tabu}

\begin{tabu}{|[2pt]p{15.68cm}|[2pt]}
	\tabucline[2pt]{-}
	\textbf{Propósito} \\
	\hline
	Consultar la información relativa a una vivienda \\
	\tabucline[2pt]{-}
\end{tabu}

\begin{tabu}{|[2pt]p{15.68cm}|[2pt]}
	\tabucline[2pt]{-}
	\textbf{Resumen} \\
	\hline
	El interesado solicita consultar la información de la vivienda deseada y se le muestra esa vista de la base de datos del sistema \\
	\tabucline[2pt]{-}
\end{tabu}
\end{center}

\subsection{Baja de vivienda}\label{CU-4.2.4}
\begin{center}
\begin{tabu}{|[2pt]p{2.5cm}|p{5cm}|p{1.5cm}|p{1.5cm}|p{1.5cm}|p{1.5cm}|[2pt]}
	\tabucline[2pt]{-}
	\textbf{Caso de uso}    & \multicolumn{4}{p{9cm}|}{\textbf{Baja de vivienda}} & \multicolumn{0}{c|[2pt]}{\cellcolor{gray!25}\textbf{CU\_08}} \\
	\hline
	\textbf{Actores}        & \multicolumn{5}{l|[2pt]}{Agente inmobiliario, arrendador} \\
	\hline
	\textbf{Tipo}           & \multicolumn{5}{l|[2pt]}{Primario y esencial} \\
	\hline
	\textbf{Referencias}    & \multicolumn{2}{l|}{RF-2.2} & \multicolumn{3}{l|[2pt]}{} \\
	\hline
	\textbf{Precondición}   & \multicolumn{5}{l|[2pt]}{Debe existir la vivienda en el sistema} \\
	\hline
	\textbf{Postcondición}  & \multicolumn{5}{l|[2pt]}{La información de la vivienda se elimina de la base de datos del sistema} \\
	\hline
	\textbf{Autor}          & Rafael Guzmán Valverde & \textbf{Fecha} & 28/03/20 & \textbf{Versión} & 1.0 \\
	\tabucline[2pt]{-}
\end{tabu}

\begin{tabu}{|[2pt]p{15.68cm}|[2pt]}
	\tabucline[2pt]{-}
	\textbf{Propósito} \\
	\hline
	Eliminar una vivienda del sistema \\
	\tabucline[2pt]{-}
\end{tabu}

\begin{tabu}{|[2pt]p{15.68cm}|[2pt]}
	\tabucline[2pt]{-}
	\textbf{Resumen} \\
	\hline
	El agente inmobiliario puede borrar cualquier vivienda del sistema y el arredador solo las que son de su propiedad \\
	\tabucline[2pt]{-}
\end{tabu}
\end{center}

\section{Gestión de contratos}
\subsection{Firma del contrato}\label{CU-4.3.1}
\begin{center}
\begin{tabu}{|[2pt]p{2.5cm}|p{5cm}|p{1.5cm}|p{1.5cm}|p{1.5cm}|p{1.5cm}|[2pt]}
	\tabucline[2pt]{-}
	\textbf{Caso de uso}    & \multicolumn{4}{p{9cm}|}{\textbf{Firma del contrato}} & \multicolumn{0}{c|[2pt]}{\cellcolor{gray!25}\textbf{CU\_09}} \\
	\hline
	\textbf{Actores}        & \multicolumn{5}{l|[2pt]}{Agente inmobiliario, arrendador y arrendatario} \\
	\hline
	\textbf{Tipo}           & \multicolumn{5}{l|[2pt]}{Primario y esencial} \\
	\hline
	\textbf{Referencias}    & \multicolumn{2}{l|}{RF-3.1} & \multicolumn{3}{l|[2pt]}{CU-08} \\
	\hline
	\textbf{Precondición}   & \multicolumn{5}{l|[2pt]}{} \\
	\hline
	\textbf{Postcondición}  & \multicolumn{5}{l|[2pt]}{El contrato se formaliza} \\
	\hline
	\textbf{Autor}          & Rafael Guzmán Valverde & \textbf{Fecha} & 28/03/20 & \textbf{Versión} & 1.0 \\
	\tabucline[2pt]{-}
\end{tabu}

\begin{tabu}{|[2pt]p{15.68cm}|[2pt]}
	\tabucline[2pt]{-}
	\textbf{Propósito} \\
	\hline
	Dejar constancia de que el contrato se ha formalizado \\
	\tabucline[2pt]{-}
\end{tabu}

\begin{tabu}{|[2pt]p{15.68cm}|[2pt]}
	\tabucline[2pt]{-}
	\textbf{Resumen} \\
	\hline
	Mediante esta función el agente inmobiliario deja constancia del contrato entre arrendador y arrendatario y da de baja temporal la vivienda \\
	\tabucline[2pt]{-}
\end{tabu}
\end{center}

\subsection{Realizar pago}\label{CU-4.3.2}
\begin{center}
\begin{tabu}{|[2pt]p{2.5cm}|p{5cm}|p{1.5cm}|p{1.5cm}|p{1.5cm}|p{1.5cm}|[2pt]}
	\tabucline[2pt]{-}
	\textbf{Caso de uso}    & \multicolumn{4}{p{9cm}|}{\textbf{Realizar pago}} & \multicolumn{0}{c|[2pt]}{\cellcolor{gray!25}\textbf{CU\_10}} \\
	\hline
	\textbf{Actores}        & \multicolumn{5}{l|[2pt]}{Arrendador y arrendatario} \\
	\hline
	\textbf{Tipo}           & \multicolumn{5}{l|[2pt]}{Primario y esencial} \\
	\hline
	\textbf{Referencias}    & \multicolumn{2}{l|}{RF-3.2} & \multicolumn{3}{l|[2pt]}{} \\
	\hline
	\textbf{Precondición}   & \multicolumn{5}{l|[2pt]}{Debe haber un contrato formalizado} \\
	\hline
	\textbf{Postcondición}  & \multicolumn{5}{l|[2pt]}{Se realiza el pago} \\
	\hline
	\textbf{Autor}          & Rafael Guzmán Valverde & \textbf{Fecha} & 28/03/20 & \textbf{Versión} & 1.0 \\
	\tabucline[2pt]{-}
\end{tabu}

\begin{tabu}{|[2pt]p{15.68cm}|[2pt]}
	\tabucline[2pt]{-}
	\textbf{Propósito} \\
	\hline
	Realizar el pago del alquiler \\
	\tabucline[2pt]{-}
\end{tabu}

\begin{tabu}{|[2pt]p{15.68cm}|[2pt]}
	\tabucline[2pt]{-}
	\textbf{Resumen} \\
	\hline
	El arrendatario paga al alquiler al arrendador, el registro de pagos queda guardado en el sistema \\
	\tabucline[2pt]{-}
\end{tabu}
\end{center}

\subsubsection{4.3.2.1: Pago con tarjeta}\label{CU-4.3.2.1}
\begin{center}
\begin{tabu}{|[2pt]p{2.5cm}|p{5cm}|p{1.5cm}|p{1.5cm}|p{1.5cm}|p{1.5cm}|[2pt]}
	\tabucline[2pt]{-}
	\textbf{Caso de uso}    & \multicolumn{4}{p{9cm}|}{\textbf{Pago con tarjeta}} & \multicolumn{0}{c|[2pt]}{\cellcolor{gray!25}\textbf{CU\_10.1}} \\
	\hline
	\textbf{Actores}        & \multicolumn{5}{l|[2pt]}{Arrendador, arrendatario} \\
	\hline
	\textbf{Tipo}           & \multicolumn{5}{l|[2pt]}{Primario y esencial} \\
	\hline
	\textbf{Referencias}    & \multicolumn{2}{l|}{} & \multicolumn{3}{l|[2pt]}{CU-10} \\
	\hline
	\textbf{Precondición}   & \multicolumn{5}{l|[2pt]}{Debe haber un contrato formalizado} \\
	\hline
	\textbf{Postcondición}  & \multicolumn{5}{l|[2pt]}{} \\
	\hline
	\textbf{Autor}          & Leire Requena Garcia & \textbf{Fecha} & 29/03/20 & \textbf{Versión} & 1.0 \\
	\tabucline[2pt]{-}
\end{tabu}

\begin{tabu}{|[2pt]p{15.68cm}|[2pt]}
	\tabucline[2pt]{-}
	\textbf{Propósito} \\
	\hline
	Realizar el pago por medio de un lector de tarjetas \\
	\tabucline[2pt]{-}
\end{tabu}

\begin{tabu}{|[2pt]p{15.68cm}|[2pt]}
	\tabucline[2pt]{-}
	\textbf{Resumen} \\
	\hline
	Se selecciona el pago con tarjeta y queda guardado en el registro \\
	\tabucline[2pt]{-}
\end{tabu}
\end{center}

\subsubsection{4.3.2.2: Pago con transferencia}\label{CU-4.3.2.2}
\begin{center}
\begin{tabu}{|[2pt]p{2.5cm}|p{5cm}|p{1.5cm}|p{1.5cm}|p{1.5cm}|p{1.5cm}|[2pt]}
	\tabucline[2pt]{-}
	\textbf{Caso de uso}    & \multicolumn{4}{p{9cm}|}{\textbf{Pago con transferencai}} & \multicolumn{0}{c|[2pt]}{\cellcolor{gray!25}\textbf{CU\_10.2}} \\
	\hline
	\textbf{Actores}        & \multicolumn{5}{l|[2pt]}{Arrendador, arrendatario} \\
	\hline
	\textbf{Tipo}           & \multicolumn{5}{l|[2pt]}{Primario y esencial} \\
	\hline
	\textbf{Referencias}    & \multicolumn{2}{l|}{} & \multicolumn{3}{l|[2pt]}{CU-10} \\
	\hline
	\textbf{Precondición}   & \multicolumn{5}{l|[2pt]}{Debe haber un contrato formalizado} \\
	\hline
	\textbf{Postcondición}  & \multicolumn{5}{l|[2pt]}{} \\
	\hline
	\textbf{Autor}          & Leire Requena Garcia & \textbf{Fecha} & 29/03/20 & \textbf{Versión} & 1.0 \\
	\tabucline[2pt]{-}
\end{tabu}

\begin{tabu}{|[2pt]p{15.68cm}|[2pt]}
	\tabucline[2pt]{-}
	\textbf{Propósito} \\
	\hline
	Realizar el pago por medio de una transferencia bancaria \\
	\tabucline[2pt]{-}
\end{tabu}

\begin{tabu}{|[2pt]p{15.68cm}|[2pt]}
	\tabucline[2pt]{-}
	\textbf{Resumen} \\
	\hline
	Se selecciona el pago por transferencia y queda guardado en el registro \\
	\tabucline[2pt]{-}
\end{tabu}
\end{center}

\subsubsection{4.3.2.3: Pago en efectivo}\label{CU-4.3.2.3}
\begin{center}
\begin{tabu}{|[2pt]p{2.5cm}|p{5cm}|p{1.5cm}|p{1.5cm}|p{1.5cm}|p{1.5cm}|[2pt]}
	\tabucline[2pt]{-}
	\textbf{Caso de uso}    & \multicolumn{4}{p{9cm}|}{\textbf{Pago en efectivo}} & \multicolumn{0}{c|[2pt]}{\cellcolor{gray!25}\textbf{CU\_10.3}} \\
	\hline
	\textbf{Actores}        & \multicolumn{5}{l|[2pt]}{Arrendador, arrendatario} \\
	\hline
	\textbf{Tipo}           & \multicolumn{5}{l|[2pt]}{Primario y esencial} \\
	\hline
	\textbf{Referencias}    & \multicolumn{2}{l|}{} & \multicolumn{3}{l|[2pt]}{CU-10} \\
	\hline
	\textbf{Precondición}   & \multicolumn{5}{l|[2pt]}{Debe haber un contrato formalizado} \\
	\hline
	\textbf{Postcondición}  & \multicolumn{5}{l|[2pt]}{} \\
	\hline
	\textbf{Autor}          & Leire Requena Garcia & \textbf{Fecha} & 29/03/20 & \textbf{Versión} & 1.0 \\
	\tabucline[2pt]{-}
\end{tabu}

\begin{tabu}{|[2pt]p{15.68cm}|[2pt]}
	\tabucline[2pt]{-}
	\textbf{Propósito} \\
	\hline
	Realizar el pago en efectivo \\
	\tabucline[2pt]{-}
\end{tabu}

\begin{tabu}{|[2pt]p{15.68cm}|[2pt]}
	\tabucline[2pt]{-}
	\textbf{Resumen} \\
	\hline
	Se selecciona el pago en efectivo (aunque es un método más incómodo) y queda guardado en el registro \\
	\tabucline[2pt]{-}
\end{tabu}
\end{center}

\section{Gestión de incidencias}
\subsection{Envío incidencia}\label{CU-4.4.1}
\begin{center}
\begin{tabu}{|[2pt]p{2.5cm}|p{5cm}|p{1.5cm}|p{1.5cm}|p{1.5cm}|p{1.5cm}|[2pt]}
	\tabucline[2pt]{-}
	\textbf{Caso de uso}    & \multicolumn{4}{p{9cm}|}{\textbf{Envío de una incidencia}} & \multicolumn{0}{c|[2pt]}{\cellcolor{gray!25}\textbf{CU\_11}} \\
	\hline
	\textbf{Actores}        & \multicolumn{5}{l|[2pt]}{Arrendatario} \\
	\hline
	\textbf{Tipo}           & \multicolumn{5}{l|[2pt]}{Primario y esencial} \\
	\hline
	\textbf{Referencias}    & \multicolumn{2}{l|}{RF-3.3} & \multicolumn{3}{l|[2pt]}{} \\
	\hline
	\textbf{Precondición}   & \multicolumn{5}{l|[2pt]}{El arrendatario debe estar registrado en el sistema} \\
	\hline
	\textbf{Postcondición}  & \multicolumn{5}{l|[2pt]}{Se envía una incidencia al sistema} \\
	\hline
	\textbf{Autor}          & Alejandro Molina Criado & \textbf{Fecha} & 28/03/20 & \textbf{Versión} & 1.0 \\
	\tabucline[2pt]{-}
\end{tabu}

\begin{tabu}{|[2pt]p{15.68cm}|[2pt]}
	\tabucline[2pt]{-}
	\textbf{Propósito} \\
	\hline
	Notificar de algúna incidencia en la vivienda \\
	\tabucline[2pt]{-}
\end{tabu}

\begin{tabu}{|[2pt]p{15.68cm}|[2pt]}
	\tabucline[2pt]{-}
	\textbf{Resumen} \\
	\hline
	El arrendatario notifica de una incidencia ocurrida en la vivienda y se notifica a la asistencia técnica \\
	\tabucline[2pt]{-}
\end{tabu}
\end{center}

\subsection{Modificar incidencia}\label{CU-4.4.2}
\begin{center}
\begin{tabu}{|[2pt]p{2.5cm}|p{5cm}|p{1.5cm}|p{1.5cm}|p{1.5cm}|p{1.5cm}|[2pt]}
	\tabucline[2pt]{-}
	\textbf{Caso de uso}    & \multicolumn{4}{p{9cm}|}{\textbf{Modificar una incidencia}} & \multicolumn{0}{c|[2pt]}{\cellcolor{gray!25}\textbf{CU\_12}} \\
	\hline
	\textbf{Actores}        & \multicolumn{5}{l|[2pt]}{Arrendatario, asistencia técnica} \\
	\hline
	\textbf{Tipo}           & \multicolumn{5}{l|[2pt]}{Primario y esencial} \\
	\hline
	\textbf{Referencias}    & \multicolumn{2}{l|}{RF-3.3} & \multicolumn{3}{l|[2pt]}{} \\
	\hline
	\textbf{Precondición}   & \multicolumn{5}{l|[2pt]}{La incidencia debe estar abierta} \\
	\hline
	\textbf{Postcondición}  & \multicolumn{5}{l|[2pt]}{El sistema guarda los cambios en la incidencia} \\
	\hline
	\textbf{Autor}          & Alejandro Molina Criado & \textbf{Fecha} & 28/03/20 & \textbf{Versión} & 1.0 \\
	\tabucline[2pt]{-}
\end{tabu}

\begin{tabu}{|[2pt]p{15.68cm}|[2pt]}
	\tabucline[2pt]{-}
	\textbf{Propósito} \\
	\hline
	Modificar la información de una incidencia \\
	\tabucline[2pt]{-}
\end{tabu}

\begin{tabu}{|[2pt]p{15.68cm}|[2pt]}
	\tabucline[2pt]{-}
	\textbf{Resumen} \\
	\hline
	El arrendatario o la asistencia técnica modifican la incidencia, pudiendo también cambiar su estado \\
	\tabucline[2pt]{-}
\end{tabu}
\end{center}

\subsection{Consulta incidencia}\label{CU-4.4.3}
\begin{center}
\begin{tabu}{|[2pt]p{2.5cm}|p{5cm}|p{1.5cm}|p{1.5cm}|p{1.5cm}|p{1.5cm}|[2pt]}
	\tabucline[2pt]{-}
	\textbf{Caso de uso}    & \multicolumn{4}{p{9cm}|}{\textbf{Consulta de incidencia}} & \multicolumn{0}{c|[2pt]}{\cellcolor{gray!25}\textbf{CU\_13}} \\
	\hline
	\textbf{Actores}        & \multicolumn{5}{l|[2pt]}{Arrendatario, asistencia técnica} \\
	\hline
	\textbf{Tipo}           & \multicolumn{5}{l|[2pt]}{Primario y esencial} \\
	\hline
	\textbf{Referencias}    & \multicolumn{2}{l|}{RF-3.3} & \multicolumn{3}{l|[2pt]}{CU-12} \\
	\hline
	\textbf{Precondición}   & \multicolumn{5}{l|[2pt]}{Debe haber una incidencia abierta o cerrada} \\
	\hline
	\textbf{Postcondición}  & \multicolumn{5}{l|[2pt]}{} \\
	\hline
	\textbf{Autor}          & Alejandro Molina Criado & \textbf{Fecha} & 29/03/20 & \textbf{Versión} & 1.0 \\
	\tabucline[2pt]{-}
\end{tabu}

\begin{tabu}{|[2pt]p{15.68cm}|[2pt]}
	\tabucline[2pt]{-}
	\textbf{Propósito} \\
	\hline
	Consultar la información y estado de una incidencia \\
	\tabucline[2pt]{-}
\end{tabu}

\begin{tabu}{|[2pt]p{15.68cm}|[2pt]}
	\tabucline[2pt]{-}
	\textbf{Resumen} \\
	\hline
	El interesado puede acceder a la información y estado de una incidencia \\
	\tabucline[2pt]{-}
\end{tabu}
\end{center}